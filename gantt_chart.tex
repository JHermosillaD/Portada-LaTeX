\documentclass[]{article}
\usepackage{tikz}
\usepackage{pgfgantt}
\usepackage[spanish]{babel}
\usepackage[spanish]{translator}

\title{Cronograma}
\author{Jesús Eduardo Hermosilla Díaz}
\date{}
\begin{document}

\maketitle

\begin{tikzpicture}
\begin{ganttchart}[
vgrid,hgrid,
title label font=\bfseries\footnotesize,
include title in canvas=false,
bar/.append style={draw=none, fill=black!60},
bar incomplete/.append style={fill=black!30},
bar progress label font=\color{white},
bar label font=\footnotesize,
]{1}{20}
\gantttitle[title label node/.append style={left= 7pt}]{Mes:}{0}
\gantttitle{Febrero}{4}
\gantttitle{Marzo}{4}
\gantttitle{Abril}{4}
\gantttitle{Mayo}{4}
\gantttitle{Junio}{4} \\
\gantttitle[title label node/.append style={left= 7pt}]{Semana:}{0}
\gantttitlelist{1,...,20}{1} \\
\ganttbar[progress=10, ]{Análisis de antecedentes}{1}{2} \\
\ganttbar[progress=20, ]{Implementación de paquetes ROS}{3}{5} \\
\ganttbar[progress=30, ]{Diseño del modelo de interacción}{6}{9} \\
\ganttbar[progress=40, ]{Simulación de escenarios realistas}{10}{12} \\
\ganttbar[progress=50, ]{Evaluación de la estrategia}{13}{14} \\
\ganttbar[progress=60, ]{Comparación del desempeño}{15}{16} \\
\ganttbar[progress=70, ]{Redacción del documento final}{17}{19} \\
\ganttbar[progress=80, ]{Presentación del trabajo recepcional}{20}{20}
\end{ganttchart}
\end{tikzpicture}

\end{document}
